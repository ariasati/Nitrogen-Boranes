\documentclass[12pt]{article}
\title{Ch 121a HW}
\author{Patryk Kozlowski}
\date{\today}
\begin{document}
\maketitle
There is a growing global economy with steep energy demands (ref article from MIT). While an energy sector that depends on renewable options is ideal, it is going to take a long time to completely phase out fossil fuels that emit carbon dioxide, a potent greenhouse gas. The technology of carbon capture offers a way to remove this carbon dioxide from the atmosphere. To advance it, we need to isolate compounds that can selectively bind CO2 for sequestration and conversion to other products, such as HCOOH. While many studies (cite here) have selectively analyzed borane complexes for capture, they do not investigate conversion to HCOOH and other products. Additionally, while irreversible CO2 binding is significant in the thermodynamic driving force behind selective CO2 capture, an understanding of the mechanism of the reversible capture of CO2 is needed. In this paper, we utilize NiBo for reversible CO2 binding. In addition, we look to elucidate the first steps of conversion of CO2 to HCOOH via the heterolytic activation of H2, done via a Frustrated Lewis Pair (FLP) type mechanism previously investigated in phosphine-borane complexes (cite). Provided the simplistic approach of using Lewis Basic sites for capturing CO2, the sp2-hybridized centers of the nitrogen present in the complex are also particularly intriguing for electrochemical reduction of CO2 to longer alkyl chain products as well. Along with the possible electrochemical pathways that these NiBo compounds introduce (cite), the lack of expensive metallic catalysts involved in their capture of CO2 means that NiBo and their derivatives remain a point of interest for CO2 capture.  
\end{document}

